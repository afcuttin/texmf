% make available title and author using the class macros
% \makeatletter
% 	\let\mytitle\@title
% 	\let\myauthor\@author
% \makeatother

\maketitle

\begin{chnglog}[semantic]
	0.1.0 & dd/mm/yyyy & Document creation.\\
	\midrule
	0.2.0 & 2020-04-03 &
		Added:
		\cref{sec:declaration}, \nameref{sec:declaration};
		\cref{sec:cla-opt}, \nameref{sec:cla-opt};
		\cref{sec:loa-pac}, \nameref{sec:loa-pac};
		\cref{sec:ava-com}, \nameref{sec:ava-com}.
	\\
	\midrule
	0.3.0 & 2021-11-23 &
		Updated:
		\cref{sec:cla-opt}, \nameref{sec:cla-opt} - added \texttt{noesslogo} option.
\end{chnglog}


\glsunsetall
\pdfbookmark[chapter]{\contentsname}{toc}
\tableofcontents
\listoffigures
\listoftables

% Glossaries
\printglossary[type=\acronymtype,title=List of acronyms]
\setglossarypreamble[symbols]{%
	The symbols here listed follow the convention stated in \cite[Sec.~1.1,][]{rfps-tech-spec}.
	Specifically, given a quantity $X$:\newline
	\begin{tabularx}{\textwidth}{lX}
		$X_N$ & denotes the nominal value of $X$; \\
		$X_m$ & denotes the minimum value of $X$; \\
		$X_M$ & denotes the maximum (peak) value of $X$; \\
		$X_{ave}$ & denotes the value of $X$, averaged on a duty cycle of an ideal square pulse with a pulse length of \SI{3.5}{\ms} and \SI{14}{\Hz} repetition rate. \\
	\end{tabularx}
	\vspace*{\baselineskip}
}
\printglossary[type=symbols,title=List of symbols]

\glsresetall

% Bibliography
\newrefcontext[labelprefix=TD-]
\printbibliography[keyword=tendoc,title=Tender Documents]

\newrefcontext[labelprefix=CD-]
\printbibliography[keyword=condoc,title=Contractor Documents]

\newrefcontext[labelprefix=RD-]
\printbibliography[notkeyword=tendoc,notkeyword=condoc,title=Reference Documents] % keyword=catb