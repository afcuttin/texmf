\documentclass[helvetica,italian,logo,notitle,totpages,utf8]{europecv2013}
\usepackage{graphicx}
\usepackage[a4paper,top=1.2cm,left=1.2cm,right=1.2cm,bottom=2.5cm]{geometry}
\usepackage[italian]{babel}
\usepackage[T1]{fontenc}
\usepackage{natbib}
\usepackage{bibentry}

%[Tutti i campi del CV sono facoltativi. Rimuovere i campi vuoti.]
\ecvname{Sostituire con Nome (i) Cognome (i)}
\ecvaddress{Sostituire con via, numero civico, codice postale, città, paese }
\ecvtelephone[Sostituire con numero telefonico]{Sostituire con telefono cellulare}
\ecvemail{Sostituire con indirizzo e-mail }
\ecvhomepage{\href{Sostituire con url a pagina personale}{Sostituire con url a pagina personale senza http://}}
\ecvlinkedin{\href{Sostituire con url al profilo linkedin}{Sostituire con url al profilo linkedin senza http://}}
\ecvgender{Indicare il sesso}
\ecvdateofbirth{gg/mm/aaaa}
\ecvnationality{Indicare la nazionalità}

\ecvfootnote{Autorizzo il trattamento dei dati personali ai sensi del \href{http://www.garanteprivacy.it/garante/doc.jsp?ID=1311248}{D. lgs. 196/03}}
%\ecvbeforepicture{\raggedleft}
%\ecvpicture[width=2.5cm]{file-immagine-eps}
%\ecvafterpicture{\ecvspace{-37mm}}

\begin{document}
\selectlanguage{italian}

\begin{europecv}
\ecvpersonalinfo[10pt]

\ecvposition{Posizione per la quale si concorre
Posizione ricoperta
Occupazione desiderata
Titolo di studio per la quale si concorre}{Sostituire con posizione per la quale si concorre / posizione ricoperta / occupazione desiderata / titolo per il quale si concorre (eliminare le voci non rilevanti nella colonna di sinistra)}

\ecvsection{Esperienza professionale}
%[Inserire separatamente le esperienze professionali svolte iniziando dalla più recente.]

\ecvworkexperience{Sostituire con date (da - a)}{Sostituire con il lavoro o posizione ricoperta}{Sostituire con il nome del datore di lavoro}{Sostituire con l'indirizzo del datore di lavoro}{Sostituire con le principali attività e responsabilità}

\ecvsection{Istruzione e formazione}
%[Inserire separatamente i corsi frequentati iniziando da quelli più recenti.]

\ecveducation{Sostituire con date (da - a)}{Sostituire con la qualifica rilasciata}{Sostituire con il nome e l'indirizzo dell'organizzazione erogatrice dell'istruzione e formazione}{
Sostituire con un elenco delle principali materie trattate o abilità acquisite}{Sostituire con il livello QEQ o altro, se conosciuto}

\ecvsection{Competenze personali}

\ecvmothertongue[20pt]{Sostituire con la lingua (e) madre}
\ecvlanguageheader
\ecvlanguage{Sostituire con la lingua}{Inserire il livello}{Inserire il livello}{Inserire il livello}{Inserire il livello}{Inserire il livello}
\ecvlastlanguage{Sostituire con la lingua}{Inserire il livello}{Inserire il livello}{Inserire il livello}{Inserire il livello}{Inserire il livello}
\ecvlanguagefooter[10pt]

\ecvitem[10pt]{Competenze comunicative}{Sostituire con le competenze comunicative possedute. Specificare in quale contesto sono state acquisite. Esempio:\par
possiedo buone competenze comunicative acquisite durante la mia esperienza di direttore vendite}
\ecvitem[10pt]{Competenze organizzative e gestionali}{Sostituire con le competenze organizzative e gestionali possedute. Specificare in quale contesto sono state acquisite. Esempio: leadership (attualmente responsabile di un team di 10 persone)}
\ecvitem[10pt]{Competenze tecniche}{Sostituire con le competenze professionali possedute non indicate altrove. Esempio:\par
buona padronanza dei processi di controllo qualità (attualmente responsabile del controllo qualità) }
\ecvitem[10pt]{Competenze informatiche}{Sostituire con le competenze informatiche possedute. Specificare in quale contesto sono state acquisite. Esempio: \par
buona padronanza degli strumenti Microsoft Office}
\ecvitem[10pt]{Altre competenze}{Sostituire con altre rilevanti competenze non ancora menzionate. Specificare in quale contesto sono state acquisite. Esempio: \par
falegnameria}

\ecvitem{Patente di guida}{Sostituire con la categoria/e della patente di guida}

\ecvsection{Ulteriori informazioni}

\ecvitem[10pt]{Pubblicazioni
Presentazioni
Progetti
Conferenze
Seminari
Riconoscimenti e premi
Appartenenza a gruppi / associazioni
Referenze}{Sostituire con rilevanti pubblicazioni, presentazioni, progetti, conferenze, seminari, riconoscimenti e premi, appartenenza a gruppi/associazioni, referenze: Rimuovere le voci non rilevanti nella colonna di sinistra.}

\ecvitem[10pt]{Dati personali}{Autorizzo il trattamento dei miei dati personali ai sensi del Decreto Legislativo 30 giugno 2003, n. 196 "Codice in materia di protezione dei dati personali (facoltativo)".}



\ecvsection{Allegati}

\ecvitem[10pt]{}{Sostituire con la lista di documenti allegati al CV. Esempio:
\begin{itemize}
\item copie delle lauree e qualifiche conseguite; 
\item attestazione di servizio;
\item attestazione del datore di lavoro.
\end{itemize}}

\bibliographystyle{plain}
\nobibliography{publications} % bib file name

\ecvsection{Publications}

\ecvitem{Pub1}{\bibentry{pub1}}
\ecvitem{Pub2}{\bibentry{pub2}}

\end{europecv}
\end{document} 
